\par En el presente informe se analizan las características acústicas de pantallas diseñadas para favorecer el aislamiento del ruido aéreo y la absorción sonora, y para ser colocadas a lo largo del viaducto del Tren Mitre, en la Ciudad Autónoma de Buenos Aire. 
La pantalla diseñada está conformada por la conjunción de  una chapa de acero galvanizado (ciega), una capa de lana de
roca, y una chapa de acero galvanizado perforada (densidad de perforación del 40\%).

\par Se realizan pruebas sobre muestras de $10 m^2$ de dichas pantallas, en las cámaras de transmisión horizontal y de reverberación del Laboratorio de Acústica y Luminotecnia (LAL) de la ciudad de La Plata. Allí, se miden niveles de presión sonora y tiempos de reverberación. Luego, se calculan parámetros a partir de los cuales se logra clasificar el material utilizado. 

\par En definitiva, se desea verificar que las pantallas diseñadas cumplan con los requerimientos establecidos acorde a las siguiente normas (Tabla~\tableref{tab:requerimientos_a_cumplir}):

\begin{table}[!h]
    \centering
    \begin{tabular}{|c|c|c|} \hline
        \textbf{Característica} & \textbf{Requerimiento} & \textbf{Normativa de aplicación}\\ \hline
        Aislamiento al ruido aéreo & Categoría B3 & UNE-EN 1793-2:2014, Anexo A\\ \hline
        Absorción sonora & Categoría A3 & UNE-EN 1793-2:2014, Anexo A \\ \hline
    \end{tabular}
    \caption{Requerimientos a cumplir}
    \label{tab:requerimientos_a_cumplir}
\end{table}


