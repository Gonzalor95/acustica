\par Para calcular el volumen del recinto se debe determinar su altura. Esto puede hacerse aplicando el \quotemarks{criterio de modos} o criterio de Bonello. Partiendo de que la existencia de modos propios es inevitable, de acuerdo con este criterio conviene elegir una relación entre las dimensiones de la sala tal que la distribución de los mismos en el eje de frecuencias sea lo más uniforme posible. De esta manera se consigue evitar concentraciones de energía en bandas.

\par Para el cálculo de frecuencias de modos de resonancia o modos propios, utilizamos la ecuación \eqref{eq::calculo_freq_resonancia}:

\begin{equation}
    \mathcolorbox{EQColor}{ f_m = \frac{c}{2} \sqrt{\Big(\frac{p}{L} \Big)^2 + \Big(\frac{q}{W} \Big)^2 + \Big(\frac{r}{H} \Big)^2 } }
    \label{eq::calculo_freq_resonancia}
\end{equation}

\par En donde:
\begin{itemize}
    \item $f_m$: Frecuencia del modo de resonancia
    \item $c$: La velocidad del sonido en $[m/s]$
    \item $p$, $q$ y $r$: Números enteros $(0, 1, 2, \ldots)$, que denotan el número de medias longitudes de onda en las 3 direcciones espaciales.
    \item $L$, $W$, $H$: Las dimensiones del recinto en $[m]$
\end{itemize}

\par De la ecuación \eqref{eq::calculo_freq_resonancia} puede notarse que dada la dependencia de la velocidad del sonido con la temperatura, trae como consecuencia que los modos también dependendan de la temperatura.

\par También mencionamos que dicha ecuación nos permite calcular todos los modos hasta la frecuencia de Schroederer. Para frecuencias superiores a esta, el especiamiento entre nodos es considerablemente pequeño y los modos resultan en un continuo produciendo que los efectos de resonancia no sean apreciables.\\

\par Se divide el espectro en tercios de octava, pues el análisis según el criterio de Bonello, satisface a la discriminación de frecuencias del oído. Se calcula y grafica la cantidad de modos por cada tercio de octava.

\par A partir de esto, el criterio establece que:

\begin{itemize}
    \item La curva de densidad de modos debe ser monótonamente creciente, o a lo sumo tener la misma cantidad de modos en dos tercios de octava sucesivos.
    \item No deben existir modos dobles y, si los hubiera, se los tolera en tercios de octava con densidad de modos mayor que cinco.
\end{itemize}

\par Dado lo expuesto, pasamos a determinar la altura de la sala utilizando el software \quotemarks{\textbf{amroc}} recomendado por la cátedra. En el mismo, se mantuvieron los datos de largo y ancho del recinto constantes, y se fue modificando la altura del recinto hasta satisfacer las condiciones y obtener una altura coherente con el fin de la sala.

\par Se obtuvo como respuesta que la altura para la sala sea de:

\begin{equation}
    \boxed{ H_{sala} = 3.4m }
\end{equation}

\par Los resultados de las dimensiones, se plasman en el cuadro \tableref{tab:medidas_de_sala}:

\begin{table}[]
\setlength\arrayrulewidth{1pt}
\arrayrulecolor{TABLEColor}
    \centering
    \begin{tabular}{|c|c|} \hline
        Altura de recinto & $H = 3.4m$ \\ \hline
        Largo de recinto  & $L = 13.9m$\\ \hline
        Ancho de recinto & $W = 7.5m$ \\ \hline
        Volumen de recinto & $V = 346.8 m^3$ \\ \hline
        Superficie de recinto & $S = 102m^2$\\ \hline
    \end{tabular}
    \caption{Medidas de la sala}
    \label{tab:medidas_de_sala}
\end{table}


