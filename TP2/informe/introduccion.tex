La finalidad es conseguir condiciones acústicas adecuadas para una buena
inteligibilidad de la palabra en una sala de conferencias. Se debe lograr que el porcentaje de palabras correctamente interpretadas por el oyente sea mayor que el 90\%.
Para ello es necesario atender dos aspectos fundamentales:

\begin{itemize}
\item El aislamiento acústico que brinde la envolvente del recinto, para protegerlo del ruido exterior y evitar que interfiera con las condiciones de audición exigidas por la actividad a desarrollar en él.
\item El acondicionamiento acústico interior, adecuando la sala al uso al que estará destinada (dimensiones, forma, materiales, etc.).

\end{itemize}

En este proyecto consideraremos que el aislamiento acústico adecuado ha sido calculado previamente. Por lo tanto, nos enfocaremos en el tratamiento acústico interior, dividiendo el trabajo en dos etapas:\\

\begin{itemize}
    \item Etapa 1: Aplicación del criterio de modos de Bonello.
    \item Etapa 2: Aplicaciones de criterios de diseño de tratamiento acústico para control de la reverberación.
\end{itemize}


\flushleft {
Contamos con los siguientes datos del proyecto: \\
$L_{sala}$ = 13,9 m \\
$A_{sala}$ = 7,5 m \\
}
 
\hspace{1cm}

